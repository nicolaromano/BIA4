\documentclass[10pt,a4paper,english]{report}
\usepackage[utf8]{inputenc}
\usepackage{amsmath}
\usepackage{amsfonts}
\usepackage{amssymb}
\usepackage{tabularx}
\usepackage{babel}
\usepackage{geometry}
\usepackage{graphicx}
\usepackage{lastpage}
\setlength{\extrarowheight}{15pt}

\pagenumbering{gobble}

\begin{document}

\section*{Marking criteria for BIA4 ICA 2 - individual report}

\begin{tabularx}{\textwidth}{|c|X|}
\hline 
\textbf{Marking range} & \textbf{Marking criteria}\\ 
\hline 
A (90-100\%) & A clear and well-written piece of work, which clearly introduces the biomedical context for the problem chosen for the group work. The report shows good evidence of critical appraisal of the literature on the subject. Excellent submissions ($>$95\%) show exceptional understanding of the problem and its context. There is clear evidence of reflection both for the student's individual contribution and for how the group worked together. There is evidence of critical evaluation of how the work could be improved, with a strong support from the current literature on the topic.\\
\hline 
B (80-89\%) & A well-written piece of work, which introduces the biomedical context for the problem chosen for the group work, with only minimal stylistic or factual errors. There is some evidence of critical appraisal of the literature on the subject. There is a somewhat limited evidence for reflection on both the student's individual contribution and how the group worked together. There some critical evaluation of how the work could be improved, with limited support from the current literature on the topic.\\ 
\hline 
C (70-79\%) &The report introduces the biomedical context for the problem chosen for the group work; there are factual errors and English can be improved. There is limited evidence of critical appraisal of the literature on the subject. Reflection on the group work is limited to either the student's individual contribution or how the group worked together and does not go into much depth.
There are some suggestions on how the work could be improved, with limited support from the current literature on the topic.\\ 
\hline 
D (60-69\%) & The report introduces the biomedical context for the problem chosen for the group work; this introduction contains several factual errors and it is not clearly written. There is very limited evidence of critical appraisal of the literature on the subject. Critical reflection on the group work is extremely limited. There are some suggestions on how the work could be improved, with little to no support from the current literature on the topic.\\ 
\hline 
E (30-59\%)& The report contains a large number of factual errors and does not clearly introduce the biomedical context for the problem chosen for the group work. There is extremely limited evidence of critical appraisal of the literature on the subject. Critical reflection on the group work is poor. There are very few suggestions on how the work could be improved, with no support from the current literature on the topic. \\ 
\hline 
F (0-29\%) & The report is difficult to understand, and shows no support from the current literature. There is hardly any reflection on the group work. There are no suggestions for future work.\\
\hline 
\end{tabularx} 

\end{document}