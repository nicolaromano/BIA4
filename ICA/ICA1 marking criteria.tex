\documentclass[10pt,a4paper,english]{report}
\usepackage[utf8]{inputenc}
\usepackage{amsmath}
\usepackage{amsfonts}
\usepackage{amssymb}
\usepackage{tabularx}
\usepackage{babel}
\usepackage{geometry}
\usepackage{graphicx}
\usepackage{lastpage}
\setlength{\extrarowheight}{15pt}

\pagenumbering{gobble}

\begin{document}

\section*{Marking criteria for BIA4 ICA 1 - group work}

\begin{tabularx}{\textwidth}{|c|X|}
\hline 
\textbf{Marking range} & \textbf{Marking criteria}\\ 
\hline 
A (90-100\%) & The group produced a software with a clear and well-motivated purpose towards solving a problem in biomedical imaging. 
Documentation is of excellent quality, written in clear and correct English, and makes it very easy to use the software. 
The Python code is well organised, clearly commented, and fully functional. Excellent submissions (over 95\%) will be robust to working with a variety of input images or able to handle incorrect user input correctly (e.g. through error messages or similar mechanisms). 
The report clearly shows the functionality of the software, by showing examples of its output (whether correct or incorrect). Excellent submissions (over 95\%) will show further analysis or quality control of the output.\\
\hline 
B (80-89\%) & The group produced a software with a clear purpose towards solving a problem in biomedical imaging. Documentation is of good quality, generally clear and/or with some minor English issues; it is generally easy to use the software. The Python code is fully functional but could be better organised; code is commented only in parts. The report shows the functionality of the software, but might have included further examples or explanations/analyses.
\\ 
\hline 
C (70-79\%) & The group produced a software with a clear purpose towards solving a problem in biomedical imaging. Documentation is understandable, but incomplete in parts or in need of improvement to help using the software. The Python code is fully functionaly or requires only minor adjustments to run; code organisation should be improved; code is only minimally commented. The report shows most of the functionality of the software.\\
\hline 
D (60-69\%) & The purpose of the software towards solving a problem in biomedical imaging is only partially clear. Documentation is incomplete or difficult to understand. It is not obvious to understand how to use the software, however the Python code is functional after some minor changes. Code is not commented, and organisation should be improved. The report shows some of the functionality of the software.\\
\hline 
E (30-59\%)& The purpose of the software towards solving a problem in biomedical imaging is only vaguely clear. Documentation is very difficult to understand and not useful in order to use the software. The Python code requires major changes to work. Code is not commented and the report is lacking in detail.\\
\hline 
F (0-29\%) & It is unclear what the purpose of the software is. Documentation is lacking or not understandable. Code is non functional. The software functionality is not shown by the report.\\
\hline 
\end{tabularx} 

\end{document}